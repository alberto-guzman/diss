\pagestyle{plain}

\begin{center}

\textbf{Deep Learning for Investigating Causal Effects with High-Dimensional Data: Analytic Tools and Applications to Educational Interventions}\\
        Alberto Guzman-Alvarez, PhD\\
        University of Pittsburgh, 2023\\
    \end{center}
       \vspace*{3\baselineskip}

Quantitative education research has the potential to be revolutionized by recent developments in machine learning. Realizing these possibilities, however, requires scholars to bridge the worlds of educational research and computer sciences. Through my dissertation, I aim to merge advances in deep learning and causal inference to enable researchers to assess program impacts using quasi-experimental methods with high-dimensional data. First, I will develop a new analytical procedure incorporating advances in Deep Learning, specifically Deep Neural Networks, to estimate propensity scores with procedures that flexibly accommodate both high-dimensional data and complex relationships between treatment selection and observable characteristics. Preliminary results suggest that these methods outperform most traditional modeling strategies, particularly when complexities (e.g., non-linearities, interactions) are present in the selection model. Second, I will incorporate this approach into a causal mediation framework that employs a propensity score-based weighting strategy to allow researchers to test potential mechanisms underlying treatment effects. In addition to the methodological contributions, my dissertation also will make substantive contributions to the applied literature. I will employ these methods to evaluate a large-scale college access intervention that offered high school students critical supports to ease their transition into college during the COVID-19 pandemic. In addition to academic papers, I will produce an open-source R package to allow applied researchers to conveniently implement my method.

\newpage