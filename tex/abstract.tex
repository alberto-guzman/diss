\pagestyle{plain}

\begin{center}
   % Wrap the title in a singlespacing environment
   \begin{singlespace}
        \textbf{Deep Learning for Investigating Causal Effects with High-Dimensional Data: Analytic Tools and Applications to Educational Interventions}
   \end{singlespace}
   
   % Add a vertical space between the title and your name
   \vspace{1\baselineskip}
   
        Alberto Guzman-Alvarez, PhD\\
        University of Pittsburgh, 2023\\
\end{center}
\vspace*{1\baselineskip}


Recent developments in machine learning have the potential to revolutionize quantitative education research. However, realizing this potential requires bridging the worlds of educational research and computer sciences. In my dissertation, I merged advances in deep learning and causal inference to enable researchers to assess program impacts using quasi-experimental methods with high-dimensional data.

My first dissertation paper proposes a new analytical procedure that incorporates deep neural networks to estimate propensity scores, which flexibly accommodate high-dimensional data and complex relationships between treatment selection and observable characteristics using propensity score weighting. In my analysis, I find that while logistic regression leads to low bias and small standard errors in the estimated average treatment effect in a low-dimensional data setting, machine learning approaches, particularly my deep neural network approach and bagged-CART, perform better in the hihg-dimensional settings.

In addition to the methodological contributions, my dissertation makes substantive contributions to the applied literature. In my second dissertation study, I evaluate a large-scale A.I. chatbot college access intervention that offered critical supports to historically and economically marginalized high school students to ease their transition into college during the COVID-19 pandemic. The study sheds light on the intervention's effectiveness and its potential for improving educational outcomes during the pandemic.

Overall, this dissertation advances the field by demonstrating the potential of machine learning and causal inference methods to advance quantitative education research. It provides a new approach for estimating propensity scores that can be used in high-dimensional settings, thereby improving the accuracy and reliability of impact assessments. The findings from the evaluation of the college access intervention offer important insights into how such programs can support students during challenging times and improve their educational outcomes, particularly for those who face systemic barriers.

\newpage